\section{电路系统的控制模型}
电路系统(仅考虑集总参数电路)的基本定律是基尔霍夫定律(KCL与KVL)。
\subsection{电阻元件}
电阻元件遵循欧姆定律,且无法贮能。
\begin{equation*}
e=Ri
\end{equation*}

\subsection{电容元件}
电容元件的定义式为$C=q/{e}$,可以贮存电场能,由定义式可以得到
\begin{equation*}
e=\frac{1}{C}\int idt
\end{equation*}

\subsection{电感元件}
电感元件可以贮存磁场能,其两端电势差与电流间的关系遵循
\begin{equation*}
e=L\frac{di}{dt}
\end{equation*}

\subsection{串联元件的传递函数}

由于元件间的负载效应,元件间的串联会导致负载效应。即,单独分析电路的一部分时,所得到的传递函数相当于负载阻抗无穷大的情况,然而实际上,能量在输出端上存在损耗。因此,串联元件的传递函数,不能直接相乘得到。

对于无负载效应,或是负载效应可以忽略不计的电路,其传递函数可以通过单个电路传递函数的乘积得到。

\subsection{复阻抗}

复阻抗$Z(s)$的定义为,元件两端电压的拉普拉斯变换$E(s)$,与通过元件电流的拉普拉斯变换$I(s)$的比,

\begin{equation*}
Z(s):=\frac{E(s)}{I(s)}
\end{equation*}

从而,电阻的复阻抗仍然为$R$,电容的复阻抗变为$1/sC$,电感的复阻抗变为$sL$。利用复阻抗,可以把所有电路元件视为具有复阻抗的线性电阻元件,直接利用基尔霍夫定律推导出传递函数,无需再逐一列些微分方程、做拉普拉斯变换,为计算带来极大的方便。

\subsection{运算放大器}

理想运算放大器的性质:
\begin{itemize}
	\item	输入电压间的电势差为0,$v_+=v_-$
	\item	输入阻抗无穷大,因此输入电流为0,即$i_+=i_-=0$
\end{itemize}



