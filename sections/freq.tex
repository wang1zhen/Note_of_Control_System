\section{频响分析法}

系统对正弦输入信号的稳态响应称为频率响应。通过改变输入信号的频率,考察系统产生的响应,可以作为分析和设计控制系统的重要依据。

正弦传递函数就是把系统传递函数中的$s$用$j\omega$代替所得到的函数,其中$\omega$为频率,单位为$rad/s$。当用$circ/s$为单位测量频率时,频率用$f$来表示,$\omega=2\pi f$。

传递函数可以表示为相位与辐角的形式

\begin{equation*}
    G(j\omega)=Me^{j\phi}=M\phase{\phi}
\end{equation*}

稳定的线性定常系统在收到正弦信号的输入后,在稳态时将会输出一个和输入信号频率相同的正弦输出信号。然而输出的幅值和相位一般和输入信号存在差异,这由正弦传递函数决定。

\begin{align*}
    |G(j \omega)|&=\left|\frac{Y(j \omega)}{X(j \omega)}\right|\\ 
    \phase{G(j \omega)}&=\phase{\frac{Y(j \omega)}{X(j \omega)}}
\end{align*}

\subsection{伯德图}

伯德图有两幅图组成,一幅是正弦传递函数幅值的对数坐标图,另一幅是相角图。两幅图的横坐标都是频率的对数。

对数幅值的表达式为$20\log\left|G(j\omega)\right|$,所得到的单位是分贝$dB$。

用MATLAB绘制伯德图的命令是

\begin{lstlisting}
    bode(num,den)
    bode(sys)
    [mag,phase,w]=bode(num,den,w)
\end{lstlisting}

得到的伯德图如图\ref{22}所示。

\begin{figure}[!ht]
    \centering
    \includegraphics[width=8cm]{figures/22.png}
    \caption{伯德图}
    \label{22}
\end{figure}

\subsection{奈奎斯特图}

奈奎斯特图也称为极坐标图,是当$\omega$由零变到无穷大时,表示在极坐标上$G(j\omega)$的幅值和相角的关系图。如图\ref{23}所示。

\begin{figure}[!ht]
    \centering
    \includegraphics[width=8cm]{figures/23.png}
    \caption{奈奎斯特图}
    \label{23}
\end{figure}

用MATLAB绘制奈奎斯特图的命令是

\begin{lstlisting}
    nyquist(num,den)
    nyquist(sys)
    [re,im,w]=nyquist(num,den,w)
\end{lstlisting}

\subsection{奈奎斯特稳定判据}

\textbf{柯西辐角原理}

对于传递函数

\begin{equation*}
    \frac{C(s)}{R(s)}=\frac{G(s)}{1+G(s)H(s)}
\end{equation*}

系统稳定的条件为特征方程

\begin{equation*}
    1+G(s)H(s)=0
\end{equation*}

的根均位于$s$左半平面。

通过分析映射后的辐角,如图\ref{24}所示,不难得到结论:在$s$平面内封闭曲线顺时针包围$F(s)$的零点一次,则映射后的封闭曲线在$F(s)$平面内顺时针包围原点一次。

\begin{figure}[!ht]
    \centering
    \includegraphics[width=8cm]{figures/24.png}
    \caption{$s$平面封闭曲线映射到$F(s)$平面的结果}
    \label{24}
\end{figure}

更一般地,则可以推广到柯西辐角原理:

\begin{equation*}
    N=Z-P
\end{equation*}

其中,$P$为$s$平面封闭曲线包围的$F(s)$极点数,$Z$为$s$平面封闭曲线包围的$F(s)$零点数,$N$为映射后$F(s)$平面封闭曲线顺时针包围原点的次数(若为负则代表逆时针包围)。

那么回到闭环传递函数与特征多项式

\begin{align*}
    T_0(s)&=\frac{C(s)G(s)}{1+C(s)G(s)}\\ 
    F(s)&=1+C(s)G(s)
\end{align*}

易得$F(s)$的零点,就对应着$T_0(s)$的极点。不妨假设$\Lambda_0(s)=C(s)G(s)$中分母的次数大于分子的次数,因此

\begin{equation*}
    \lim_{|s|\rightarrow\infty}F(s)=1
\end{equation*}

\begin{figure}[!ht]
    \centering
    \includegraphics[width=8cm]{figures/25.png}
    \caption{$s$平面内封闭曲线的选取}
    \label{25}
\end{figure}

选取如图\ref{25}的封闭曲线,假设在虚轴上不存在零极点,则该曲线包围了右边平面$F(s)$全部的零极点。因此,如果我们知道了$\Lambda_0(s)$右半平面的极点数$P$,即可以通过观察$\Lambda_0(s)$奈奎斯特图中围绕点$(-1,j0)$点的次数$N$,就能得到$F(s)$的零点数$Z$。

因此,奈奎斯特稳定判据可以简单地表述为:如果开环传递函数$G(s)H(s)$在右半平面内有$P$个极点,则当$\omega$从$-\infty$变到$\infty$时,$G(j\omega)H(j\omega)$逆时针包围$(-1,j0)$点$P$次时,闭环系统稳定。

\subsection{$G(s)H(s)$含有位于虚轴上零极点的特殊情况}

当在虚轴上有开环传递函数的零极点时,就需要对$s$平面上的封闭曲线进行一定的改造,如图\ref{26}所示。

\begin{figure}[!ht]
    \centering
    \includegraphics[width=8cm]{figures/26.png}
    \caption{$s$平面封闭曲线的改造}
    \label{26}
\end{figure}

例如:考虑开环传递函数

\begin{equation*}
    G(s)H(s)=\frac{K}{s(Ts+1)}
\end{equation*}

在$G(s)H(s)$平面上,对应$s=j0+$和$s=j0-$的点分别为$-j\infty$和$+j\infty$。在半径$\epsilon$的半圆轨迹上,如图\ref{27}所示,可以表示为

\begin{equation*}
    s=\epsilon e^{i\theta}\quad \theta=-90^\circ\rightarrow+90^\circ
\end{equation*}

\begin{figure}[!ht]
    \centering
    \includegraphics[width=8cm]{figures/27.png}
    \caption{改造后曲线}
    \label{27}
\end{figure}

对应的$G(s)H(s)$变为

\begin{equation*}
    G\left(\varepsilon \mathrm{e}^{j \theta}\right) H\left(\varepsilon \mathrm{e}^{j \theta}\right)=\frac{K}{\varepsilon \mathrm{e}^{j \theta}}=\frac{K}{\varepsilon} \mathrm{e}^{-j \theta}
\end{equation*}

可以得到对应的$G(s)H(s)$平面的封闭曲线如图\ref{28}所示。由于$s$右半平面没有极点,$G(s)H(s)$轨迹也不包围$(-1,j0)$点,因此$1+G(s)H(s)$没有位于右半平面的零点,系统稳定。

\begin{figure}[!ht]
    \centering
    \includegraphics[width=8cm]{figures/28.png}
    \caption{$G(s)H(s)$平面上的封闭曲线}
    \label{28}
\end{figure}

\subsection{相对稳定性分析}

为了表示稳定系统的稳定程度,可以定义一系列相对稳定性参数。

\textbf{相位裕量}

开环传递函数的幅值$|G(j\omega)|$等于1时,此处的相角$\Phi$加$180^\circ$即为相位裕量

\begin{equation*}
    \gamma=180^\circ + \Phi\quad\mbox{Phase Margin, }M_f
\end{equation*}

\textbf{增益裕量}

在相角等于$-180^\circ$的频率上,幅值$|G(j\omega)|$的倒数称为增益裕量

\begin{equation*}
    K_g=\frac{1}{|G(j\omega)|}
\end{equation*}

一般以分贝表示

\begin{equation*}
    K_g(dB)=20\log K_g=-20\log |G(j\omega)|\quad\mbox{Gain Margin, }M_g
\end{equation*}

\textbf{灵敏度峰值}

以$-1,j0$为圆心,与Nyquist轨迹不相交的最大的圆的半径的倒数。

\begin{equation*}
    \frac{1}{\eta}=\frac{1}{\min|1+\Lambda_0(j\omega)|}\quad\mbox{Sensitivity Peak}
\end{equation*}

图\ref{29}展示了相对稳定性参数在伯德图和奈奎斯特图中的意义。

\begin{figure}[!ht]
    \centering
    \includegraphics[width=8cm]{figures/29.png}
    \caption{相位裕量和增益裕量}
    \label{29}
\end{figure}

一般而言,要保证系统具有良好的稳定性,要求相位裕量$PM>30^\circ$,增益裕量$GM>15dB$,灵敏度峰值$SP<4$。

\subsection{频率响应法设计控制系统(频域补偿)}

开环频率响应的伯德图中,可以大致分为低频区、中频区(增益交界频率附近的区域)和高频区。

低频区代表了系统的稳态特性以及对输入信号的追踪能力;中频区能体现系统的瞬态响应和相对稳定性;高频区则表征了系统的鲁棒性和拒绝干扰信号的能力。

\textbf{闭环带宽}

闭环带宽的定义是,当频率落在$[0,\omega_{BW}]$的范围内时,闭环传递函数的对数幅值$\geq-3dB$,即

\begin{equation*}
    20\log\left|\frac{\Lambda(j\omega)}{1+\Lambda(j\omega)}\right|\geq-3dB, \forall \omega\in[0,\omega_{BW}]
\end{equation*}

一般来说,我们估计闭环带宽是开环交界频率的$1\sim2$倍,

\begin{equation*}
    \omega_{BW}\in[\omega_c,2\omega_c]
\end{equation*}

\subsection{基础补偿器}

\textbf{超前校正}

超前校正装置的传递函数是

\begin{gather*}
    C(s)=K\frac{\tau_z s+1}{\tau_p s+1}=K_c\alpha \frac{Ts+1}{\alpha Ts+1}\\ 
    0<\tau_p<\tau_z, 0<\alpha<1
\end{gather*}

超前校正装置的零极点都是稳定的,且零点在极点的右侧(零点更接近原点)。

\begin{figure}[!ht]
    \centering
    \includegraphics[width=8cm]{figures/30.png}
    \caption{超前校正装置Nyquist图}
    \label{30}
\end{figure}

图\ref{30}所示的是$K_c=1$时的Nyquist图。可以看到,最大相位超前角$\phi_m$在$\omega=\omega_M$时取到,因此

\begin{gather*}
    \sin\phi_m=\frac{1-\alpha}{1+\alpha}
\end{gather*}

\begin{figure}[!ht]
    \centering
    \includegraphics[width=8cm]{figures/31.png}
    \caption{超前校正装置Bode图}
    \label{31}
\end{figure}

图\ref{31}是对应的Bode图,可以发现$\omega_m$是两个转角频率的中点,

\begin{equation*}
    \omega_m=\frac{1}{\sqrt\alpha T}
\end{equation*}

\textbf{基于频率响应法的超前校正}

超前校正能够产生足够大的相位超前角,补偿相位裕量。

超前校正的步骤:

\begin{itemize}
    \item 设计超前校正装置:
    
    \begin{equation*}
        C(s)=K_c\alpha\frac{Ts+1}{\alpha Ts+1}=K\frac{Ts+1}{\alpha Ts+1}
    \end{equation*}

    根据给定的静态误差常数要求,可以确定增益$K$。

    \item 绘出$KG(s)$的Bode图,求出相位裕量。
    \item 确定需要增加的相位超前角。考虑到超前校正装置会使得增益交接频率增加,导致相位裕量减少,因此需要在这个基础上额外增加$5^\circ\sim 12^\circ$来确定$\phi_m$。
    \item 通过$\phi_m$求出衰减因子$\alpha$。接着找到$KG(s)$幅值对应为$-20\log(1/\sqrt{\alpha})$的频率作为新的增益交接频率$\omega_c$。根据$\omega_c=1/(\sqrt\alpha T)$确定$T$。
    \item 零点位置$1/T$,极点位置$1/\alpha T$,增益$K_c=K/\alpha$。
    \item 检查增益裕量是否满足要求。如果不满足,则改变零极点位置,重复上述步骤。
\end{itemize}

\textbf{滞后校正}

滞后校正的传递函数是

\begin{gather*}
    C(s)=K\frac{\tau_z s+1}{\tau_p s+1}=K_c\beta\frac{Ts+1}{\beta T s+1}\\ 
    0<\tau_z<\tau_p, \beta>1
\end{gather*}

滞后校正装置的零极点也是稳定的,极点位于零点的右边。

\begin{figure}[!ht]
    \centering
    \includegraphics[width=8cm]{figures/32.png}
    \caption{滞后校正装置Nyquist图}
    \label{32}
\end{figure}

\begin{figure}[!ht]
    \centering
    \includegraphics[width=8cm]{figures/33.png}
    \caption{滞后校正装置Bode图}
    \label{33}
\end{figure}

滞后校正装置的Nyquist图和Bode图如图\ref{32}和图\ref{33}所示。

\textbf{基于频率响应法的滞后校正}

滞后校正的主要目的是把交界频率向左推,从而使得交界频率处的相位裕量增大。

滞后校正的步骤:

\begin{itemize}
    \item 设计滞后校正装置
    
    \begin{equation*}
        C(s)=K_c\beta\frac{Ts+1}{\beta Ts+1}=K\frac{Ts+1}{\beta Ts+1}
    \end{equation*}

    根据给定的静态误差常数要求,可以确定增益$K$。

    \item 检查仅通过增益$K$得到的系统$KG(s)$是否满足相位裕量和增益裕量的性能指标。若不满足,则寻找一个新的频率,使得该点上的相角等于$-180^\circ$加要求的相位裕量。要求的相位裕量应等于指定的相位裕量额外增加$5^\circ\sim 12^\circ$来补偿滞后校正装置带来的相位滞后。选择这个频率作为新的增益交界频率。
    \item 滞后校正装置的极点和零点应远低于新的增益交界频率,因此选取转角频率$\omega=1/T$比交界频率低一倍到十倍频程。
    \item 确定复制曲线在新的交界频率下降到$0dB$所需的衰减量,该衰减量等于$-20\log\beta$,从而确定$\beta$。
    \item 确定增益$K_c=K/\beta$。
    \item 检查是否满足要求,否则重复上述步骤。
\end{itemize}

\textbf{超前-滞后校正}

暂时先空着